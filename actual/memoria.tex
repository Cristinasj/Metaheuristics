\documentclass[12pt, spanish]{article}
\newcommand\tab[1][1cm]{\hspace*{#1}}
\usepackage[spanish]{babel}
\selectlanguage{spanish}
\usepackage{natbib}
\usepackage{url}
\usepackage[utf8x]{inputenc}
\usepackage{graphicx}
\graphicspath{{images/}}
\usepackage{parskip}
\usepackage{fancyhdr}
\usepackage{vmargin}
\usepackage{multirow}
\usepackage{pgfplots}
\usepackage{float}
\usepackage{algorithm}
\usepackage{algorithmic}
\usepackage{chngpage}
\usepackage{enumitem}
\usepackage{amsmath}
\usepackage{algorithm}
\usepackage{algpseudocode}

\usepackage{subcaption}

\usepackage{hyperref}
\usepackage[
    type={CC},
    modifier={by-nc-sa},
    version={4.0},
]{doclicense}

\hypersetup{
    colorlinks=true,
    linkcolor=blue,
    filecolor=magenta,      
    urlcolor=cyan,
}

% para codigo
\usepackage{listings}
\usepackage{xcolor}



%% configuración de listings

\definecolor{listing-background}{HTML}{F7F7F7}
\definecolor{listing-rule}{HTML}{B3B2B3}
\definecolor{listing-numbers}{HTML}{B3B2B3}
\definecolor{listing-text-color}{HTML}{000000}
\definecolor{listing-keyword}{HTML}{435489}
\definecolor{listing-identifier}{HTML}{435489}
\definecolor{listing-string}{HTML}{00999A}
\definecolor{listing-comment}{HTML}{8E8E8E}
\definecolor{listing-javadoc-comment}{HTML}{006CA9}

\lstdefinestyle{eisvogel_listing_style}{
  language         = c++,
%$if(listings-disable-line-numbers)$
%  xleftmargin      = 0.6em,
%  framexleftmargin = 0.4em,
%$else$
  numbers          = left,
  xleftmargin      = 0em,
 framexleftmargin = 0em,
%$endif$
  backgroundcolor  = \color{listing-background},
  basicstyle       = \color{listing-text-color}\small\ttfamily{}\linespread{1.15}, % print whole listing small
  breaklines       = true,
  frame            = single,
  framesep         = 0.19em,
  rulecolor        = \color{listing-rule},
  frameround       = ffff,
  tabsize          = 4,
  numberstyle      = \color{listing-numbers},
  aboveskip        = 1.0em,
  belowskip        = 0.1em,
  abovecaptionskip = 0em,
  belowcaptionskip = 1.0em,
  keywordstyle     = \color{listing-keyword}\bfseries,
  classoffset      = 0,
  sensitive        = true,
  identifierstyle  = \color{listing-identifier},
  commentstyle     = \color{listing-comment},
  morecomment      = [s][\color{listing-javadoc-comment}]{/**}{*/},
  stringstyle      = \color{listing-string},
  showstringspaces = false,
  escapeinside     = {/*@}{@*/}, % Allow LaTeX inside these special comments
  literate         =
  {á}{{\'a}}1 {é}{{\'e}}1 {í}{{\'i}}1 {ó}{{\'o}}1 {ú}{{\'u}}1
  {Á}{{\'A}}1 {É}{{\'E}}1 {Í}{{\'I}}1 {Ó}{{\'O}}1 {Ú}{{\'U}}1
  {à}{{\`a}}1 {è}{{\'e}}1 {ì}{{\`i}}1 {ò}{{\`o}}1 {ù}{{\`u}}1
  {À}{{\`A}}1 {È}{{\'E}}1 {Ì}{{\`I}}1 {Ò}{{\`O}}1 {Ù}{{\`U}}1
  {ä}{{\"a}}1 {ë}{{\"e}}1 {ï}{{\"i}}1 {ö}{{\"o}}1 {ü}{{\"u}}1
  {Ä}{{\"A}}1 {Ë}{{\"E}}1 {Ï}{{\"I}}1 {Ö}{{\"O}}1 {Ü}{{\"U}}1
  {â}{{\^a}}1 {ê}{{\^e}}1 {î}{{\^i}}1 {ô}{{\^o}}1 {û}{{\^u}}1
  {Â}{{\^A}}1 {Ê}{{\^E}}1 {Î}{{\^I}}1 {Ô}{{\^O}}1 {Û}{{\^U}}1
  {œ}{{\oe}}1 {Œ}{{\OE}}1 {æ}{{\ae}}1 {Æ}{{\AE}}1 {ß}{{\ss}}1
  {ç}{{\c c}}1 {Ç}{{\c C}}1 {ø}{{\o}}1 {å}{{\r a}}1 {Å}{{\r A}}1
  {€}{{\EUR}}1 {£}{{\pounds}}1 {«}{{\guillemotleft}}1
  {»}{{\guillemotright}}1 {ñ}{{\~n}}1 {Ñ}{{\~N}}1 {¿}{{?`}}1
  {…}{{\ldots}}1 {≥}{{>=}}1 {≤}{{<=}}1 {„}{{\glqq}}1 {“}{{\grqq}}1
  {”}{{''}}1
}
\lstset{style=eisvogel_listing_style}


\usepackage[default]{sourcesanspro}

\setmarginsrb{2 cm}{1 cm}{2 cm}{2 cm}{1 cm}{1.5 cm}{1 cm}{1.5 cm}

\title{Práctica 3:\\
APC - Búsquedas por Trayectorias  \hspace{0.05cm} }                           
\author{Cristina Sánchez Justicia}                             
\date{\today}                                           

\renewcommand*\contentsname{hola}

\makeatletter
\let\thetitle\@title
\let\theauthor\@author
\let\thedate\@date
\makeatother

\pagestyle{fancy}
\fancyhf{}
\rhead{\theauthor}
\lhead{\thetitle}
\cfoot{\thepage}

\begin{document}

%%%%%%%%%%%%%%%%%%%%%%%%%%%%%%%%%%%%%%%%%%%%%%%%%%%%%%%%%%%%%%%%%%%%%%%%%%%%%%%%%%%%%%%%%

\begin{titlepage}
    \centering
    \vspace*{0.3 cm}
    \includegraphics[scale = 0.50]{ugr.png}\\[0.7 cm]
    %\textsc{\LARGE Universidad de Granada}\\[2.0 cm]   
    \textsc{\large 3º CSI 2022/23 - Grupo 1}\\[0.5 cm]                \textsc{\large Grado en Ingeniería Informática}\\[0.5 cm]              
    \rule{\linewidth}{0.2 mm} \\[0.2 cm]
    { \huge \bfseries \thetitle}\\
	Algoritmos: 
	1-NN, RELIEF, BL, AGE-BLX, AGE-CA, BMB, ES, ILS, ILS-ES, VNS \\
    \rule{\linewidth}{0.2 mm} \\[1 cm]
    
    \begin{minipage}{0.4\textwidth}
        \begin{flushleft} \large
            \emph{Autor:}\\
            \theauthor\\ 
			 \emph{DNI:}\\
            77689772G
            \end{flushleft}
            \end{minipage}~
            \begin{minipage}{0.4\textwidth}
            \begin{flushright} \large
            \emph{Asignatura: \\
            Metaheurísticas}   \\     
            \emph{Correo:}\\
            cristina@correo.ugr.es           
        \end{flushright}
    \end{minipage}\\[0.5cm]
  
    {\large \thedate}\\[0.5cm]
    {\url{https://github.com/cristinasj/MH/}}
 	
    \vfill
    
\end{titlepage}

%%%%%%%%%%%%%%%%%%%%%%%%%%%%%%%%%%%%%%%%%%%%%%%%%%%%%%%%%%%%%%%%%%%%%%%%%%%%%%%%%%%%%%%%%

\tableofcontents
\pagebreak

%%%%%%%%%%%%%%%%%%%%%%%%%%%%%%%%%%%%%%%%%%%%%%%%%%%%%%%%%%%%%%%%%%%%%%%%%%%%%%%%%%%%%%%%%


\section{Introducción}
\subsection{Qué es GWO}
El Grey Wolf Optimization (GWO), también conocido como Algoritmo de Optimización del Lobo 
Gris, es un algoritmo de optimización bioinspirado que se basa en el comportamiento 
social de los lobos grises en la naturaleza. Fue propuesto por primera vez por Seyedale 
Mirjalili en 2014.

\subsection{Cómo funciona GWO}
El GWO se inspira en la jerarquía social y las interacciones de los lobos grises en su 
búsqueda de presas. En la manada de lobos, hay una estructura de dominancia donde existen 
líderes (alfa, beta y delta) y seguidores. Los líderes guían la búsqueda y los seguidores 
los imitan para encontrar presas de manera eficiente.
\newline 
También existen los lobos omega, que no tienen dominancia sobre ningún otro. 

\subsection{Implementación}
En el algoritmo GWO, las soluciones candidatas se representan como "lobos" en el espacio de 
búsqueda. Cada lobo tiene una posición en el espacio y una función de aptitud que indica 
qué tan buena es su solución. Los lobos líderes representan las mejores soluciones 
encontradas hasta el momento mientras que los lobos omega representan el resto de las 
soluciones. 
\newline
Durante la ejecución del algoritmo, los lobos exploran el espacio de búsqueda en busca de 
mejores soluciones. Los líderes influyen en el movimiento de los seguidores mediante la 
actualización de sus posiciones y velocidades. Se utiliza una función de actualización basada 
en la jerarquía social de los lobos para determinar los nuevos movimientos.

\subsection{Resumen}
El Grey Wolf Optimization (GWO) es un algoritmo de optimización bioinspirado que se 
basa en el comportamiento social de los lobos grises para encontrar soluciones eficientes a 
problemas de optimización. Utiliza una jerarquía de líderes y seguidores para guiar la búsqueda 
y encontrar mejores soluciones en el espacio de búsqueda.

\section{Explotación y exploración}
En el algoritmo Grey Wolf Optimization (GWO), tanto la exploración como la explotación se 
llevan a cabo durante la búsqueda de soluciones en el espacio de búsqueda. Estas dos 
estrategias son fundamentales para encontrar soluciones óptimas.\\
\newline
La exploración se refiere a la capacidad del algoritmo para buscar en diferentes regiones 
del espacio de búsqueda en busca de soluciones prometedoras. Durante la exploración, los 
lobos se mueven con relativa libertad por el espacio de búsqueda, lo que les permite explorar nuevas 
áreas y descubrir soluciones potencialmente mejores. Esta fase es esencial para evitar que el 
algoritmo quede atrapado en óptimos locales subóptimos y para ampliar el espacio de búsqueda.\\
\newline
La explotación, por otro lado, se centra en aprovechar las soluciones prometedoras encontradas 
hasta el momento para refinar y mejorar aún más esas soluciones. Durante la explotación, los 
lobos se centran en las mejores soluciones encontradas (los líderes) y ajustan sus posiciones y 
velocidades para acercarse a esas soluciones. Esto implica un movimiento más dirigido y enfocado 
hacia las regiones del espacio de búsqueda que contienen soluciones de alta calidad.

\section{Modelación matemática}

\section{Pseudocódigo}

\section{Adaptación a APC}

\section{Experimentos}

\section{Comparación con la práctica}

\section{Propuesta de mejora}


\end{document}